% vim: set spell spelllang=en tw=100 et sw=4 sts=4 :

\documentclass{llncs}

% \usepackage{showframe}

\usepackage{microtype}
\usepackage{complexity}                % \P, \NP etc
\usepackage{tikz}                      % For pretty pictures
\usepackage{amsmath}                   % \operatorname
\usepackage{amsfonts}                  % \mathcal
\usepackage{amssymb}                   % \nexists
\usepackage{hyperref}                  % clicky links
\usepackage{cleveref}                  % no need to type Figure etc
\usepackage{enumitem}                  % better lists
\usepackage{gnuplot-lua-tikz}          % graphs
\usepackage{placeins}                  % floatbarrier
\usepackage[ruled,vlined]{algorithm2e} % algorithms (after cleverref!)

\usetikzlibrary{arrows, shadows, calc, positioning, decorations, decorations.pathreplacing, patterns}

% lncs style
\crefname{algocf}{Algorithm}{Algorithms}
\Crefname{algocf}{Algorithm}{Algorithms}
\crefname{figure}{Fig.}{Figs.}
\Crefname{figure}{Fig.}{Figs.}
\crefname{table}{Table}{Tables}
\Crefname{table}{Table}{Tables}
\crefname{proposition}{Proposition}{Propositions}
\Crefname{proposition}{Proposition}{Propositions}

\newcommand{\lineref}[1]{line~\ref{#1}}

\title{A Parallel, Backjumping Subgraph Isomorphism Algorithm using Supplemental Graphs}

\author{Ciaran McCreesh\thanks{This work was supported by the Engineering and Physical Sciences
Research Council [grant number EP/K503058/1]} \and Patrick Prosser}
\institute{University of Glasgow, Glasgow, Scotland \\ \email{c.mccreesh.1@research.gla.ac.uk}, \\
    \email{patrick.prosser@glasgow.ac.uk}}

\begin{document}

\maketitle

\begin{abstract}
    The subgraph isomorphism problem involves finding a pattern graph inside a target graph.  We
    present a new bit- and thread-parallel constraint-based search algorithm for the problem, and
    experiment on a wide range of standard benchmark instances to demonstrate its effectiveness.  We
    introduce the idea of supplemental graphs, to create implied constraints. We use a low-overhead,
    lazy variation of conflict directed backjumping which interacts safely with parallel search, and
    a counting-based all-different propagator which is better suited to our large domains than the
    conventional matching approach.
\end{abstract}

\section{Introduction}

The subgraph isomorphism family of problems involve ``finding a copy of'' a pattern graph inside a
larger target graph; applications include bioinformatics \cite{Bonnici:2013}, chemistry
\cite{Regin:1995}, computer vision \cite{Damiand:2011,Solnon:2015}, law enforcement
\cite{Coffman:2004}, model checking \cite{Sevegnani:2015}, and pattern recognition
\cite{Conte:2004}.  These problems have natural constraint programming models: we have a variable
for each vertex in the pattern graph, with the vertices of the target graph being the domains. The
exact constraints vary depending upon which variation of the problem we are studying (which we
discuss in the following section), but generally there are rules about preserving adjacency, and an
all-different constraint across all the variables.

This constraint-based search approach dates back to works by Ullmann \cite{Ullmann:1976} and
McGregor \cite{McGregor:1979}, and was improved upon in the LV \cite{Larrosa:2002} and VF2
\cite{Cordella:2004} algorithms.  More recently, ILF \cite{Zampelli:2010}, LAD \cite{Solnon:2010}
and SND \cite{Audemard:2014} are algorithms which take a ``deep thinking'' approach, using strong
inference at each stage of the search.  This is powerful, but we observe that when LAD or SND fail
to find a solution in time, it is often because they do not get to explore much of the search space.
Sometimes LAD and SND make less than one recursive call per second---this is often true with larger
target graphs, where strong inference becomes particularly expensive.  This motivates our
alternative approach: on the same hardware, we will be making between ten thousand and a million
recursive calls per core per second on non-trivial instances.  The main features of our algorithm
are:

\begin{enumerate}
    \item We introduce supplemental graphs, which generalise some of the ideas in SND. The key idea
        is that a subgraph isomorphism $i : P \hookrightarrow T$ induces a subgraph isomorphism
        $F(i) : F(P) \hookrightarrow F(T)$, for certain functors $F$.  This is used to generate
        implied constraints: we may now look for a mapping which is simultaneously a subgraph
        isomorphism between several carefully selected pairs of graphs.

    \item We use weaker inference than LAD and SND: we do not achieve or maintain arc consistency.
        We introduce a cheaper, counting-based all-different propagator which has better scalability
        for large target graphs, and which has very good constant factors on modern hardware thanks
        to bitset encodings.

    \item We describe a clone-comparing variation of conflict-directed backjumping, which does not
        require explicit conflict sets. We note that an all-different propagator can produce reduced
        conflict sets, which can enable more backjumping.

    \item We use thread-parallel preprocessing and search, to make better use of modern multi-core
        hardware. We explain how parallel search may interact safely with backjumping. We use
        explicit, non-randomised work stealing to offset poor early heuristic choices during search.
\end{enumerate}

\noindent Although weaker propagation and backjumping have fallen out of fashion in general for
constraint programming, here this approach usually pays off. In \cref{section:experiments} we show
that over a large collection of instances commonly used to compare subgraph isomorphism algorithms,
our solver is the single best.

\section{Definitions, Notation, and a Proposition}

Throughout, our graphs are finite, undirected, and do not have multiple edges between pairs of
vertices, but may have loops (an edge from a vertex to itself). We write $\operatorname{V}(G)$ for
the vertex set of a graph $G$, and $\operatorname{N}(G, v)$ for the neighbours of a vertex $v$ in
$G$ (that is, the vertices adjacent to $v$).  The \emph{degree} of $v$ is the cardinality of its set
of neighbours. The \emph{neighbourhood degree sequence} of $v$, denoted $\operatorname{S}(G, v)$, is
the sequence consisting of the degrees of every neighbour of $v$, from largest to smallest.  A
vertex is \emph{isolated} if it has no neighbours. By $v \sim_G w$ we mean vertex $v$ is adjacent to
vertex $w$ in graph $G$. We write $G[V]$ for the subgraph of $G$ induced by a set of vertices $V$.

A \emph{non-induced subgraph isomorphism} is an injective mapping $i : P \hookrightarrow T$ from a
graph $P$ to a graph $T$ which preserves adjacency---that is, if $v \sim_{P} w$ then we require
$i(v) \sim_{T} i(w)$ (and thus if $v$ has a loop, then $i(v)$ must have a loop). The
\emph{non-induced subgraph isomorphism problem} is to find such a mapping from a given pattern graph
$P$ to a given target graph $T$.  (The \emph{induced subgraph isomorphism problem} additionally
requires that if $v \not\sim_{P} w$ then $i(v) \not\sim_{T} i(w)$; we discuss only the non-induced
version in this paper. Both variants are \NP-complete.)

If $R$ and $S$ are sequences of integers, we write $R \preceq S$ if there exists a subsequence of
$S$ with length equal to that of $R$, such that each element in $R$ is less than or equal to the
corresponding element in $S$.  For a set $U$ and element $v$, we write $U - v$ to mean $U \setminus
\{ v \}$, and $U + v$ to mean $U \cup \{ v \}$.

A \emph{path} in a graph is a sequence of distinct vertices, such that each successive pair of
vertices are adjacent; we also allow a path from a vertex to itself, in which case the first and
last vertices in the sequence are the same (and there is a \emph{cycle}). The \emph{distance}
between two vertices is the length of a shortest path between them. We write $G^d$ for the graph
with vertex set $\operatorname{V}(G)$, and edges between $v$ and $w$ if the distance between $v$ and
$w$ in $G$ is at most $d$.  We introduce the notation $G^{\left[c, l\right]}$ for the graph with
vertex set $\operatorname{V}(G)$, and edges between vertices $v$ and $w$ (not necessarily distinct)
precisely if there are at least $c$ paths of length exactly $l$ between $v$ and $w$ in $G$. The
following proposition may easily be verified (by observing that injectivity means paths are
preserved):

\begin{proposition}\label{proposition:supplemental}
    Let $i : P \hookrightarrow T$ be a subgraph isomorphism. Then $i$ is also
    \begin{enumerate}[itemindent=1em,topsep=0pt,itemsep=0.2pt]
        \item a subgraph isomorphism $i^d : P^d \hookrightarrow T^d$ for any $d \ge 1$, and
        \item a subgraph isomorphism $i^{\left[c, l\right]} : P^{\left[c, l\right]} \hookrightarrow
            T^{\left[c, l\right]}$ for any $c, l \ge 1$.
    \end{enumerate}
\end{proposition}

\noindent The (contrapositive of the) first of these facts is used by SND, which dynamically
performs distance-based filtering during search. We will instead use the second fact, at the top of
search, to generate implied constraints.

\section{A New Algorithm}

\Cref{algorithm:sip} describes our approach.  We begin (\lineref{line:enough}) with a simple check
that there are enough vertices in the pattern graph for an injective mapping to exist.  We then
(\lineref{line:isolated}) discard isolated vertices in the pattern graph---such vertices may be
greedily assigned to any remaining target vertices after a solution is found. This reduces the
number of variables which must be copied when branching.  Next we construct the supplemental graphs
(\lineref{line:buildsup}) and initialise domains (\lineref{line:initdomains}). We then
(\lineref{line:initalldiff}) use a counting-based all-different propagator to reduce these domains
further. Finally, we perform a backtracking search (\lineref{line:search}). Each of these steps is
elaborated upon below.

\begin{algorithm}[h]
\DontPrintSemicolon
\nl $\FuncSty{nonInducedSubgraphIsomorphism}$ (Graph $\mathcal{P}$, Graph $\mathcal{T}$) $\rightarrow$ Bool \;
\nl \Begin{
    \nl \lIf{$|\operatorname{V}(\mathcal{P})| > |\operatorname{V}(\mathcal{T})|$}{$\KwSty{return}~\KwSty{false}$\label{line:enough}}
    \nl Discard isolated vertices in $\mathcal{P}$\label{line:isolated} \;
    \nl $L \gets \FuncSty{createSupplementalGraphList}(\mathcal{P}, \mathcal{T})$ \label{line:buildsup}\;
    \nl $D \gets \FuncSty{init}(\operatorname{V}(\mathcal{P}), \operatorname{V}(\mathcal{T}), L)$
    \label{line:initdomains}\;
    \nl \lIf{$\FuncSty{countingAllDifferent}(D) \ne \textnormal{Success}$}{$\KwSty{return}~\KwSty{false}$\label{line:initalldiff}}
    \nl $\KwSty{return}~\FuncSty{search}(L, D) = \textnormal{Success}$ \label{line:search} \;
}

\caption{A non-induced subgraph isomorphism algorithm}
\label{algorithm:sip}
\end{algorithm}

\subsection{Preprocessing and Initialisation}

Following \cref{proposition:supplemental}, in \cref{algorithm:supplemental} we construct a sequence
of supplemental graph pairs from our given pattern and target graph. We will then search for a
mapping which is simultaneously a mapping from each pattern graph in the sequence to its paired
target graph---this gives us implied constraints, leading to additional filtering during search.

Our choice of supplemental graphs is somewhat arbitrary. We observe that distances of greater than 3
rarely give additional filtering power, and constructing $G^{\left[c, 4\right]}$ is computationally
very expensive (for unbounded $l$, the construction is \NP-hard). Checking $c > 3$ is also rarely
beneficial. Our choices work reasonably well in general on the wide range of benchmark instances we
consider, but can be expensive for trivial instances---thus there is potential room to improve the
algorithm by better selection on an instance by instance basis \cite{Malitsky:2014}.

\begin{algorithm}[h]
\DontPrintSemicolon
\nl $\FuncSty{createSupplementalGraphList}$ (Graph $\mathcal{P}$, Graph $\mathcal{T}$) $\rightarrow$ GraphPairs \;
\nl \Begin{
    \nl $\begin{aligned}\KwSty{return}~\Big[
            &(\mathcal{P},\ \mathcal{T}),
            &&(\mathcal{P}^{\left[1,2\right]},\ \mathcal{T}^{\left[1,2\right]}),
            &&(\mathcal{P}^{\left[2,2\right]},\ \mathcal{T}^{\left[2,2\right]}),
            &&(\mathcal{P}^{\left[3,2\right]},\ \mathcal{T}^{\left[3,2\right]}), \\
            &&&(\mathcal{P}^{\left[1,3\right]},\ \mathcal{T}^{\left[1,3\right]}),
            &&(\mathcal{P}^{\left[2,3\right]},\ \mathcal{T}^{\left[2,3\right]}),
            &&(\mathcal{P}^{\left[3,3\right]},\ \mathcal{T}^{\left[3,3\right]})
    \Big]\end{aligned}$ \;
}
\caption{Supplemental graphs for \cref{algorithm:sip}}
\label{algorithm:supplemental}
\end{algorithm}

\Cref{algorithm:init} is responsible for initialising domains. We have a variable for each vertex in
the (original) pattern graph, with each domain being the vertices in the (original) target graph. It
is easy to see that a vertex of degree $d$ in the pattern graph $P$ may only be mapped to a vertex in
the target graph $T$ of degree $d$ or higher: this allows us to perform some initial filtering. By
extension, we may use compatibility of neighbourhood degree sequences: $v$ may only be mapped to $w$
if $\operatorname{S}(P, v) \preceq \operatorname{S}(T, w)$ \cite{Zampelli:2010}.  Because any
subgraph isomorphism $P \hookrightarrow T$ is also a subgraph isomorphism $F(P) \hookrightarrow
F(T)$ for any of our supplemental graph constructions $F$, we may further restrict initial domains
by considering only the intersection of filtered domains using each supplemental graph pair
individually (\lineref{line:filter}). At this stage, we also enforce the ``loops must be mapped to loops'' constraint.

\begin{algorithm}[h]
\DontPrintSemicolon
\nl $\FuncSty{init}$ (Vertices $V$, Vertices $R$, GraphPairs $L$) $\rightarrow$ Domains \;
\nl \Begin{
    \nl \Repeat{\nl$R$ \textnormal{is unchanged}}{
        \nl \ForEach{$v \in V$}{
            \nl $D_v \gets \bigcap_{(P,\,T) \in L} \left\{ w \in R :~
                v\underset{P}{\sim}v \Rightarrow w\underset{T}{\sim}w \wedge \operatorname{S}(P, v) \preceq
            \operatorname{S}(T[R], w) \right\}$ \label{line:filter} \;
        }
    \nl $R \gets \bigcup_{v \in V} D_v$ \label{line:rred} \;
}
\nl $\KwSty{return}~D$ \;
}
\caption{Variable initialisation for \cref{algorithm:sip}}
\label{algorithm:init}
\end{algorithm}

Following this filtering, some target vertices may no longer appear in any domain, in which case $R$
will be reduced on \lineref{line:rred}. If this happens, we iteratively repeat the domain
construction, but do not consider any target vertex no longer in $R$ when calculating degree
sequences. (Note that for performance reasons, we do not recompute supplemental graphs when this
occurs.)

\subsection{Search and Inference}

\Cref{algorithm:search} describes our recursive search procedure. If every variable has already been
assigned, we succeed (\lineref{line:success}). Otherwise, we pick a variable (\lineref{line:branch})
to branch on by selecting the variable with smallest domain, tiebreaking on static degree only in
the original pattern graph (we tried other variations, including using supplemental graphs for
calculating degree, and domain over degree, but none gave an overall improvement).  For each value
in its domain in turn, ordered by static degree in the target graph \cite{Geelen:1992}, we try
assigning that value to the variable (\lineref{line:assign}). If we do not detect a failure, we
recurse (\lineref{line:recursesearch}).

\begin{algorithm}[h]
\DontPrintSemicolon
\SetKwSwitch{Switch}{Case}{Other}{case}{of}{}{otherwise}{endcase}{endsw}
\nl $\FuncSty{search}$ (GraphPairs $L$, Domains $D$) $\rightarrow$ Fail $F$ \KwSty{or} Success \;
\nl \Begin{
    \nl \lIf {$D = \emptyset$}{$\KwSty{return}~\textnormal{Success}$\label{line:success}}
    \nl $D_v \gets \textnormal{a domain in $D$ with minimum size, tiebreaking on static degree in}~\mathcal{P}$ \label{line:branch} \;
    \nl $F \gets \{ v \}$ \label{line:fcurrent} \;
    \nl \ForEach {$v' \in D_v~\textnormal{ordered by static degree in}~\mathcal{T}$\label{line:parforeach}}{
        \nl $D' \gets \FuncSty{clone}(D)$ \;
        \nl \Switch{$\FuncSty{assign}(L, D', v, v')$\label{line:assign}}{
            \nl \lCase{\textnormal{Fail $F'$~\KwSty{then}}}{$F \gets F \cup F'$\label{line:fassign}}
            \nl \Case{\textnormal{Success~\KwSty{then}}}{
                \nl \Switch{$\FuncSty{search}(L, D' - D_v)$\label{line:recursesearch}}{
                    \nl \lCase{\textnormal{Success~\KwSty{then}}}{$\KwSty{return}~\textnormal{Success}$}
                    \nl \Case{\textnormal{Fail $F'$~\KwSty{then}}\label{line:searchfailed}}{
                        \nl \lIf{$\nexists\,w \in F'~\KwSty{such that}~D_w \ne D'_w$}{$\KwSty{return}~\textnormal{Fail}~F'$\label{line:backjump}}
                        \nl $F \gets F \cup F'$ \label{line:fsub} \;
                    }
                }
            }
        }
    }
    \nl $\KwSty{return}~\textnormal{Fail}~F$ \;
}
\caption{Recursive search for \cref{algorithm:sip}}
\label{algorithm:search}
\end{algorithm}

The assignment and recursive search both either indicate success, or return a nogood set of
variables $F$ which cannot all be instantiated whilst respecting assignments which have already been
made. This information is used to prune the search space: if a subproblem search has failed
(\lineref{line:searchfailed}), but the current assignment did not remove any value from any of the
domains involved in the discovered nogood (\lineref{line:backjump}), then we may ignore the current
assignment and backtrack immediately.  In fact this is simply conflict-directed backjumping
\cite{Prosser:1993a} in disguise: rather than explicitly maintaining conflict sets to determine
culprits (which can be costly when backjumping does nothing \cite{Bessiere:1996,Gent:2010}), we
lazily create the conflict sets for the variables in $F'$ as necessary by comparing $D$ before the
current assignment with the $D'$ created after it. Finally, as in backjumping, if none of the
assignments are possible, we return with a nogood of the current variable (\lineref{line:fcurrent})
combined with the union of the nogoods of each failed assignment (\lineref{line:fassign}) or
subsearch (\lineref{line:fsub}).

For assignment and inference, \Cref{algorithm:assign} gives the value $v'$ to the domain $D_v$
(\lineref{line:assigngets}), and then infers which values may be eliminated from the remaining
domains.  Firstly, no other domain may now be given the value $v'$ (\lineref{line:removev}).
Secondly, for each supplemental graph pair, any domain for a vertex adjacent to $v$ may only be
mapped to a vertex adjacent to $v'$ (\lineref{line:adj}). If any domain gives a wipeout, then we
fail with that variable as the nogood (\lineref{line:wipeout}).

\begin{algorithm}[h]
\DontPrintSemicolon
\nl $\FuncSty{assign}$ (GraphPairs $L$, Domains $D$, Vertex $v$, Vertex $v'$) $\rightarrow$ Fail $F$ \KwSty{or} Success \;
\nl \Begin{
    \nl $D_v \gets \{ v' \}$ \label{line:assigngets} \;
\nl \ForEach{$D_w \in D - D_v$}{
    \nl $D_w \gets D_w - v'$ \label{line:removev} \;
    \nl \ForEach{$(P,\,T) \in L$}{
        \nl \lIf{$v \sim_P w$}{$D_w \gets D_w \cap \operatorname{N}(T, v')$\label{line:adj}}
    }
    \nl \lIf{$D_w = \emptyset$}{$\KwSty{return}~\textnormal{Fail}~\{ w \}$\label{line:wipeout}}
}
\nl $\KwSty{return}~\FuncSty{countingAllDifferent}(D)$ \;
}
\caption{Variable assignment for \cref{algorithm:assign}}
\label{algorithm:assign}
\end{algorithm}

To enforce the all-different constraint, it suffices to remove the assigned value from every other
domain, as we did in \lineref{line:removev}. However, it is often possible to do better. We can
sometimes detect that an assignment is impossible even if values remain in each variable's domain
(if we can find a set of $n$ variables whose domains include strictly less than $n$ values between
them, which we call a \emph{failed Hall set}), and we can remove certain variable-value assignments
that we can prove will never occur (if we can find a set of $n$ variables whose domains include only
$n$ values between them, which we call a \emph{Hall set}, then those values may be removed from the
domains of any other variable). The canonical way of doing this is to use R\'egin's matching-based
propagator \cite{Regin:1994}.

However, matching-based filtering is expensive and may do relatively little, particularly when
domains are large, and the payoff may not always be worth the cost. Various approaches to offsetting
this cost while maintaining the filtering power have been considered \cite{Gent:2008}. Since we are
not maintaining arc consistency in general, we instead use an intermediate level of inference which
is not guaranteed to identify every Hall set: this can be thought of as a heuristic towards the
matching approach. This is described in \cref{algorithm:cad}.

\begin{algorithm}[h]
\DontPrintSemicolon
\nl $\FuncSty{countingAllDifferent}$ (Domains $D$) $\rightarrow$ Fail $F$ \KwSty{or} Success \;
\nl \Begin{
\nl $F \gets \emptyset$ \;
\nl $(H,\,A,\,n) \gets (\emptyset,\,\emptyset,\,0)$ \;
\nl \ForEach{$D_v \in D$ \textnormal{from smallest cardinality to largest\label{line:eachdomain}}}{
    \nl $F \gets F + v$ \;
    \nl $D_v \gets D_v \setminus H$ \label{line:elimhall} \;
    \nl $(A,\,n) \gets (A \cup D_v,\,n + 1)$ \label{line:acc} \;
    \nl \lIf{$D_v = \emptyset~\KwSty{or}~|A| < n$}{$\KwSty{return}~\textnormal{Fail}~F$\label{line:failhall}}
    \nl \lIf{$|A| = n$}{$(H,\,A,\,n) \gets (H \cup A,\,\emptyset,\,0)$\label{line:hall}}
}
\nl $\KwSty{return}~\textnormal{Success}$ \;
}
\caption{Counting-based all-different propagation}
\label{algorithm:cad}
\end{algorithm}

The algorithm works by performing a linear pass over each domain in turn, from smallest cardinality
to largest (\lineref{line:eachdomain}). The $H$ variable contains the union of every Hall set
detected so far; initially it is empty. The $A$ set accumulates the union of domains seen so far,
and $n$ contains the number of domains contributing to $A$. For each new domain we encounter, we
eliminate any values present in previous Hall sets (\lineref{line:elimhall}). We then add that
domain's values to $A$ and increment $n$ (\lineref{line:acc}). If we detect a failed Hall set, we
fail (\lineref{line:failhall}).  If we detect a Hall set, we add those values to $H$, and reset $A$
and $n$, and keep going (\lineref{line:hall}).

It is important to note that this approach may fail to identify some Hall sets, if the initial
ordering of domains is imperfect. However, the algorithm runs very quickly in practice: the
sorting step is $\mathcal{O}(v \log{} v)$ (where $v$ is the number of remaining variables), and the
loop has complexity $\mathcal{O}(v d)$ (where $d$ is the cost of a bitset operation over a target
domain, which we discuss below). We validate this trade-off experimentally in the following section.

In case a failure is detected, the $F$ set of nogoods we return need only include the variables
processed so far, not every variable involved in the constraint. This is because an all-different
constraint implies an all-different constraint on any subset of its variables. A smaller set of
nogoods can increase the potential for backjumping (and experiments verified that this is beneficial
in practice).

We have been unable to find this algorithm described elsewhere in the literature, although a sort-
and counting-based approach has been used to achieve bounds consistency \cite{Puget:1998} (but our
domains are not naturally ordered) and as a preprocessing step \cite{Quimper:2005}. Bitsets (which
we discuss below) have also been used to implement the matching algorithm \cite{VanKessel:2012}.

\subsection{Bit- and Thread-Parallelism}

The use of bitset encodings for graph algorithms to exploit hardware parallelism dates back to at
least Ullmann's algorithm \cite{Ullmann:1976}, and remains an active area of research
\cite{SanSegundo:2007,Ullmann:2011}. We use bitsets here: our graphs are stored as arrays of bit
vectors, our domains are bit vectors, the neighbourhood intersection in \cref{algorithm:assign} is a
bitwise-and operation, the unions in \cref{algorithm:search} and \cref{algorithm:cad} are bitwise-or
operations, and the cardinality check in \cref{algorithm:cad} is a population count (this is a
single instruction in modern CPUs).  In addition to the SIMD-like parallelism from bitset encodings,
we observed two opportunities for multi-core thread parallelism in the algorithm:

\paragraph{Graph and domain construction} We may parallelise the outer $\FuncSty{for}$ loops
involved in calculating neighbourhood degree sequences and in initialising the domains of variables
in \cref{algorithm:init}.  Similarly, constructing each supplemental graph in
\cref{algorithm:supplemental} involves an outer $\KwSty{for}$ loop, iterating over each vertex in
the input graph. These loops may also be parallelised, with one caveat: we must be able to add edges
to (but not remove edges from) the output graph safely, in parallel. This may be done using an
atomic ``or'' operation.

\paragraph{Search} Viewing the recursive calls made by the $\FuncSty{search}$ function in
\cref{algorithm:search} as forming a tree, we may explore different subtrees in parallel. The key
points are:

\begin{enumerate}
    \item We do not know in advance whether the $\KwSty{foreach}$ loop (\cref{algorithm:search}
        \lineref{line:parforeach}) will exit early (either due to a solution being found, or
        backjumping). Thus our parallelism is speculative: we make a single thread always preserve
        the sequential search order, and use any additional threads to precompute subsequent entries
        in the loop which \emph{might} be used. This may mean we get no speedup at all, if our
        speculation performs work which will not be used.

    \item The $\KwSty{search}$ function, parallelised without changes, could attempt to exit early
        due to backjumping. We rule out this possibility by refusing to pass knowledge to the left:
        that is, we do not allow speculatively-found backjumping conditions to change the return
        value of $\KwSty{search}$). This is for safety \cite{Trienekens:1990} and reproducibility:
        value-ordering heuristics can alter the performance of unsatisfiable instances when
        backjumping, and allowing parallelism to select a different backjump set could lead to an
        absolute slowdown \cite{Prosser:1993b}. To avoid this possibility, when a backjump condition
        is found, we \emph{must} cancel any speculative work being done to the right of its
        position, and \emph{cannot} cancel any ongoing work to the left.  This means that unlike in
        conventional backtracking search without learning, we should \emph{not} expect a linear
        speedup for unsatisfiable instances.

        (In effect we are treating the $\KwSty{foreach}$ loop as a parallel fold, so that a subtree
        does not depend upon items to its left. Backjumping conditions are left-zero elements
        \cite{Lobachev:2012}, although we do not have a unique zero.)

    \item If any thread finds a solution, we \emph{do} succeed immediately, even if this involves
        passing knowledge to the left. If there are multiple solutions, this can lead to a parallel
        search finding a different solution to the one which would be found sequentially---since the
        solution we find is arbitrary, this is not genuinely unsafe. However, this means we could
        witness a superlinear (greater than $n$ from $n$ threads) speedup for satisfiable instances
        \cite{deBruin:1995}.

    \item For work stealing, we explicitly prioritise subproblems highest up and then furthest left
        in the search tree. This is because we expect our value-ordering heuristics to be weakest
        early on in search \cite{Harvey:1995}, and we use parallelism to offset poor choices early
        on in the search \cite{Chu:2009,McCreesh:2015}.
\end{enumerate}

\section{Experimental Evaluation}\label{section:experiments}

Our algorithm was implemented\footnote{source code is available at
\url{https://github.com/ciaranm/parasols}} in C++ using C++11 native threads, and was compiled using
GCC 4.9.0.  We performed our experiments on a machine with dual Intel Xeon E5-2640 v2 processors
(for a total of 16 cores, and 32 hardware threads via hyper-threading), running Scientific Linux
6.6.  For the comparison with SND in the following section, we used Java HotSpot 1.8.0\_11.

For evaluation, we used the same families of benchmark instances that were used to evaluate LAD
\cite{Solnon:2010} and SND \cite{Audemard:2014}. The ``LV'' family \cite{Larrosa:2002} contains
graphs with various interesting properties from the Stanford Graph Database, and the 793
pattern/target pairs give a mix of satisfiable and unsatisfiable queries. The ``SF'' family contains
100 scale-free graph pairs, again mixing satisfiable and unsatisfiable queries. The remainder of
these graphs come from the Vflib database \cite{Cordella:2004}: the ``BVG'' and ``BVGm'' families
are bounded degree graphs (540 pairs all are satisfiable), ``M4D'' and ``M4Dr'' are four-dimensional
meshes (360 pairs, all satisfiable), and the ``r'' family is randomly generated (270 pairs, all
satisfiable).  We expanded this suite with 24 pairs of graphs representing image pattern queries
\cite{Damiand:2011} (which we label ``football''), and 200 randomly selected pairs from each of a
series of 2D image (``images'') and 3D mesh (``meshes'') graph queries \cite{Solnon:2015}. The
largest number of vertices is 900 for a pattern and 5,944 for a target, and the largest number of
edges is 12,410 for a pattern and 34,210 for a target; some of these graphs do contain loops. All
2,487 instances are publicly available in a simple text
format\footnote{\url{http://liris.cnrs.fr/csolnon/SIP.html}}.

\subsection{Comparison with Other Solvers}

We compare our implementation against the Abscon 609 implementation of SND (which is written in
Java) \cite{Audemard:2014}, Solnon's C implementation of LAD \cite{Solnon:2010}, and the VFLib C
implementation of VF2 \cite{Cordella:2004}. (The versions of each of these solvers we used could
support loops in graphs correctly.)

Note that SND is not inherently multi-threaded, but the Java 8 virtual machine we used for testing
makes use of multiple threads for garbage collection even for sequential code. On the one hand,
this could be seen as giving SND an unfair advantage. However, nearly all modern CPUs are multi-core
anyway, so one could say that it is everyone else's fault for not taking advantage of these extra
resources. We therefore present both sequential (from a dedicated implementation, \emph{not} a
threaded implementation running with only a single thread) and threaded results for our algorithm.

\begin{figure}[tb]
    \centering
    \input{gen-graph-cumulative}

    \caption{Cumulative number of benchmark instances solved within a given time, for different
    algorithms: at time $t$, the value is the size of the set of instances whose runtime is at most
    $t$ for that algorithm. Parallel results are using 32 threads on a 16 core hyper-threaded
    system.}
    \label{figure:cumulative}
\end{figure}

\begin{figure}[p]
    \centering
    \input{gen-graph-best-other}

    \caption{Above, our sequential runtime compared to the virtual best other sequential solver,
        for each benchmark instance; below, the same, with our parallel runtimes and including
        parallel solvers. For points below the diagonal line, ours is the best solver for this
        instance; for points above the diagonal, the point colour indicates the best solver.}
    \label{figure:best-other}
\end{figure}

In \cref{figure:cumulative} we show the cumulative performance of each algorithm. The value of
the line at a given time for an algorithm shows the total number of instances which, individually,
were solved in at most that amount of time. Our sequential implementation beats VF2 for times over
0.2s, LAD for times over 0.6s, and always beats SND. Our parallel implementation beats VF2 for times
over 0.06s, LAD for times over 0.02s, and always beats SND. Parallelism gives us an overall benefit
from 12ms onwards.

\Cref{figure:best-other} presents an alternative perspective of these results. Each point represents
an instance, and the shape of a point shows its family. For the $y$ position for an instance, we use
our sequential (top graph) or parallel (bottom graph) runtime. For the $x$ position, we use the
runtime from the virtual best other solver; the colour of a point indicates which solver this is.
For any point below the $x=y$ diagonal line, we are the best solver. A limit of $10^8$\,ms was
used---points along the outer axes represent timeouts.

Although overall ours is the single best solver, VF2 is stronger on trivial instances. This is not
surprising: we must spend time constructing supplemental graphs. Thus it may be worth using VF2 as a
presolver, if short runtimes for trivial instances is desirable---this may be the case in graph
database systems where many trivial queries must be run \cite{Giugno:2013} (although these systems
could cache the supplemental graphs for targets). These results also suggest potential scope for
algorithm portfolios, or instance-specific configuration: for example, we could omit or use
different supplemental graphs in some cases.

\subsection{Parallelism}

\begin{figure}[p]
    \centering
    \input{gen-graph-speedup}

    \caption{The effects of parallelism, using 32 threads on a 16 core hyper-threaded
    system.  Each point is a problem instance; points below the diagonal line indicate a speedup.}
    \label{figure:speedup}
\end{figure}

\begin{figure}[p]
    \centering
    \input{gen-graph-backjumping}

    \caption{The effects of backjumping. Each point is one benchmark instance; points below the
    diagonal line indicate a speedup.}
    \label{figure:bj}
\end{figure}

\begin{figure}[p]
    \centering
    \input{gen-graph-fad}

    \caption{Above, the improvement to the search space size which would be given by R\'egin's
    matching-based all-different propagator. Below, the improvement given by using counting
    all-different rather than simple deletion. Each point is one benchmark instance; the point style
    shows the benchmark family. Points below the diagonal line indicate a reduction in the search space
    size.}
    \label{figure:fad}
\end{figure}

\Cref{figure:speedup} shows, for each instance, the speedup obtained from parallelism. Except at
very low sequential runtimes, we see a reasonable general improvement.  For some satisfiable
instances, we see strongly superlinear speedups. These instances are exceptionally hard problems
\cite{Smith:1997}: we would have found a solution quickly, except for a small number of wrong turns
at the top of search. Our work stealing strategy was able to avoid strong commitment to early
value-ordering heuristic choices, providing an alternative to using more complicated sequential
search strategies to offset this issue. (Some of these results were also visible in
\cref{figure:best-other}, where we timed out on satisfiable instances which another solver found
trivial.)

Some non-trivial satisfiable instances exhibited a visible slowdown. This is because we were using
32 software threads, to match the advertised number of hardware hyper-threads, but our CPUs only
have 16 ``real'' cores between them. For these instances parallelism did not reduce the amount of
work required to find a solution, but did result in a lower rate of recursive calls per second on
the sequential search path---this is similar to the risk of introducing a slower processing element
to a cluster \cite{Trienekens:1990}. Even when experimenting with 16 threads, we sometimes observed
a small slowdown due to worse cache and memory bus performance, and due to the overhead of modifying
the code to allow for work stealing (recall that we are benchmarking against a dedicated sequential
implementation).

In a small number of cases, we observe low speedups for non-trivial unsatisfiable instances. These
are from cases where backjumping has a substantial effect on search, making much of our speculative
parallelism wasted effort. (Additionally, if cancellation were not to be used, some of these
instances would exhibit large absolute slowdowns.)

\subsection{Effects of Backjumping}

In \cref{figure:bj} we show the benefits of backjumping: points below the diagonal line indicate an
improvement to runtimes from backjumping. Close inspection shows that backjumping usually at least
pays for itself, or gives a slight improvement. (This was not the case when we maintained conflict
sets explicitly: there, the overheads lead to a small average slowdown.)

For a few instances, backjumping makes an improvement of several orders of magnitude. The effects
are most visible for some of the LV benchmarks, which consist of highly structured graphs. This
mirrors the conclusions of Chen and Van Beek \cite{Chen:2001}, who saw that ``adding CBJ to a
backtracking algorithm \ldots can (still) speed up the algorithm by several orders of magnitude on
hard, structured problems''.  Real-world graphs often have unexpected structural properties which
are not present in random instances \cite{MacIntyre:1998,Slater:2014}, so we consider backjumping to
be worthwhile.

\subsection{Comparing All-Different Propagators}

We now justify our use of the counting all-different propagator. In the top half of
\cref{figure:fad} we show the benefits to the size of the search space that would be gained if we
used R\'egin's algorithm at every step instead of our counting propagator (cutting search off after
$10^9$ search nodes). We see that for most graph families, there would be little to no benefit even
if there was no additional performance cost. Only in a small portion of the LV graphs do we see a
gain (and in one case, due to dynamic variable ordering, there is a penalty).

Thus, either our counting propagator is nearly always as good as matching, or neither propagator
does very much in this domain. In the bottom half of \cref{figure:fad} we show the benefits to the
size of the search space that are gained from using counting, rather than simply deleting a value
from every other domain on assignment. The large number of points below the diagonal line confirm
that going beyond simple value deletion for all-different propagation is worthwhile.

\section{Conclusion}

Going against conventional wisdom, we saw that replacing strong inference with cheaper surrogates
could pay off, and that backjumping could be implemented cheaply enough to be beneficial. We also
saw parallelism give a substantial benefit. This was true even for relatively low runtimes, due to
us exploiting parallelism for pre-processing as well as for search.  Parallel backjumping has only
been given limited attention \cite{Conrad:1994,Habbas:1997,Cope:2000}.  However, a simple approach
has worked reasonably well here (in contrast to stronger learning systems, where successes in
parallelism appear to be rare).

There is also plenty of scope for extensions of and improvement to our algorithm. We have yet to
deeply investigate the possibility of constructing domain- or instance-specific supplemental graphs.
Nor did we discuss directed graphs or induced isomorphisms: supplemental graphs can be taken further
for these variations of the problem.  In particular, composing transformations for induced
isomorphisms would allow us to reason about ``paths of non-edges'', which may be very helpful.
Finally, we have yet to consider exploiting the symmetries and dominance relations which we know are
present in many graph instances.

\section*{Acknowledgements}

The authors wish to thank Christine Solnon for discussions, providing the graph instances and the
LAD implementation, Christophe Lecoutre for discussion and the SND implementation, and Lars Kotthoff
and Alice Miller for their comments.

\FloatBarrier
\bibliographystyle{splncs}
\bibliography{paper}

\end{document}

